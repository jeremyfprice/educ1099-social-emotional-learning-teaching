%!TEX program = luatex

%\documentclass[two-side]{tufte-handout}
\documentclass{article}
\usepackage[margin=1in]{geometry}
\usepackage{fontspec}
\usepackage{hyperref}
\usepackage{amsmath}
\usepackage{titlesec}
%\usepackage{framed}
\usepackage{arydshln}
\usepackage{fontawesome}
%\usepackage{pdfpages}
\usepackage{csquotes}
\MakeOuterQuote{"}

\titleformat
{\part} % command
[display] % shape
{\bfseries\Large\itshape} % format
{Story No. \ \thechapter} % label
{0.5ex} % sep
{
    \rule{\textwidth}{1pt}
    \vspace{1ex}
    \centering
} % before-code
[
\vspace{-0.5ex}%
\rule{\textwidth}{0.3pt}
] % after-code

% Set up the images/graphics package
\usepackage{graphicx}
\setkeys{Gin}{width=\linewidth,totalheight=\textheight,keepaspectratio}
\graphicspath{{syllabus-img/}}

\title{Social and Emotional Approaches to Learning and Teaching}
\author{Dr. Jeremy Price}
\date{Fall 2015}  % if the \date{} command is left out, the current date will be used

% The following package makes prettier tables.  We're all about the bling!
\usepackage{booktabs}

% The units package provides nice, non-stacked fractions and better spacing
% for units.
\usepackage{units}

% The fancyvrb package lets us customize the formatting of verbatim
% environments.  We use a slightly smaller font.
\usepackage{fancyvrb}
\fvset{fontsize=\normalsize}

% Small sections of multiple columns
\usepackage{multicol}

% Provides paragraphs of dummy text
\usepackage{lipsum}

\usepackage{microtype}

\usepackage{newpxtext,newpxmath}
\useosf % old-style figures in text, not in math

\newcommand{\gentopic}[1]{\begin{center}\faKey \textsf{#1}\end{center}}
\newcommand{\tabpq}{\faQuestionCircle\medspace\textit{Priming Questions}}
\newcommand{\tabread}{\faBook\medspace\textit{Readings}}
\newcommand{\tabperformance}{\faTasks\medspace\textit{Performances}}
\newcommand{\tabdt}{\faCalendar\medspace\textit{Week Of}}
\newcommand{\tabcheckin}{\faCheckSquareO\medspace\textit{Check-In Performances}}
\newcommand{\tabbreak}{\begin{center}\faAsterisk\faAsterisk\faAsterisk\\\end{center}}
\newcommand{\specialweek}[1]{\begin{center}\textbf{\faBullhorn\medspace Special Week: #1 \medspace\faBullhorn}\end{center}}
\newenvironment{tabsched}
	{\small
	\begin{tabular}{p{1.5in}p{4.5in}}
	\midrule}
	{\midrule
	\end{tabular}
	\normalsize}

\newenvironment{specweek}
	{\begin{center}
		\fontseries{b} \faBullhorn \medspace Special Week: }
		{\medspace \faBullhorn \fontseries{m}
	\end{center}}

\newcommand{\weekone}{August 17-21}
\newcommand{\weektwo}{August 24-28}
\newcommand{\weekthree}{August 31-September 4}
\newcommand{\weekfour}{September 7-11}
\newcommand{\weekfive}{September 14-18}
\newcommand{\weeksix}{September 21-25}
\newcommand{\weekseven}{September 28-October 2}
\newcommand{\weekeight}{October 5-9}
\newcommand{\laborday}{Labor Day on Monday, September 7 (no class)}
\newcommand{\roshhashanah}{Rosh Hashanah on Monday, September 14 (class via VoiceThread)}
\newcommand{\yomkippur}{Yom Kippur on Wednesday, September 23 (class via VoiceThread)}
\newcommand{\finisemester}{\begin{center}\large\textbf{\faFlagCheckered Last Day of Class, Wednesday, October 7 \faFlagCheckered}\normalsize\end{center}}

\newcommand{\listmon}{\item[Monday] \hfill \\}
\newcommand{\listwed}{\item[Wednesday] \hfill \\}
\newenvironment{daywu}
	{\textbf{\underline{Warm-Up:}} \hfill \\
	\begin{itemize}}
	{\end{itemize}}
\newenvironment{dayact}
	{\textbf{\underline{Activities:}} \hfill \\
	\begin{itemize}}
	{\end{itemize}}
\newenvironment{dayref}
	{\textbf{\underline{Reflection:}} \hfill \\
	\begin{itemize}}
	{\end{itemize}}
\newenvironment{weeksched}
	{\noindent
	\begin{description}}
	{\end{description}
	\newpage}

% Set up the spacing using fontspec features
%\renewcommand\allcapsspacing[1]{{\addfontfeature{LetterSpace=15}#1}}
%\renewcommand\smallcapsspacing[1]{{\addfontfeature{LetterSpace=10}#1}}
\setmainfont{Alice in Wonderland}
\setsansfont{Gill Sans MT}

\begin{document}
\section{Unit 1: Introducing Social and Emotional Approaches to Teaching and Learning}

\gentopic{Values and attitudes influence the educational process, are brought in by both students and educators,\\and can be developed through intentional efforts.}


\begin{tabsched}
	\tabdt & \weekone \\
	\midrule
	\tabread & The Need for Social and Emotional Learning (\url{http://bit.ly/1WauXyA}) \\
	& Why Emotional Learning May Be As Important As The ABCs \\
	& (\url{http://n.pr/1OZuvO1}) \\
	& How to Be Emotionally Intelligent (\url{http://nyti.ms/1hrH6it}) \\
	& Nonacademic Skills are Key To Success. But What Should We Call Them?  (\url{http://n.pr/1PeOtFh}) \\
	\midrule
	\tabcheckin & \textbf{What Do I Notice?} Due Friday, August 21 \\
\end{tabsched}
\begin{weeksched}

\listmon
\begin{daywu}
	\item Purposeful First Day Exercise
		\begin{itemize}
			\item Name on one side
			\item What kind of educator/person do you want to be?
			\item What kind of impact do you want to have?
		\end{itemize}
	\item Pair-Share, and then share with class. Additional question: How do you want this class to help you with these answers?
\end{daywu}
\begin{dayact}
	\item Discuss brief overview of class and review syllabus
	\item Read poem, discuss in hevrutah
	\begin{itemize}
		\item Why is "ideal" important?
		\item How does this poem relate to learning?
		\item How does this poem relate to knowing?
	\end{itemize}
\end{dayact}
\begin{dayref}
	\item What's one thing you've found surprising today in class?
\end{dayref}

\listwed
\begin{daywu}
	\item Second-Day Graffiti
\end{daywu}
\begin{dayact}
	\item Pause for questions about the course, requirements, or due dates
	\item "Lecture" on Framework
\end{dayact}
\begin{dayref}
	\item Anonymous remaining questions on note cards
\end{dayref}
\textsf{Test test test}

\end{weeksched}
\tabbreak

\begin{tabsched}
	\tabdt & \weektwo \\
	\midrule
	\tabread & How Emotions Affect Learning (\url{http://bit.ly/1TivmL9})\\
	& The Science of Inside Out (\url{http://nyti.ms/1TivSsn})\\
	& Four Lessons from Inside Out to Discuss With Kids (\url{http://bit.ly/1KY4ONg})\\
	\midrule
	\tabperformance & \textbf{48-Hour Social-Emotional Journal} Due Monday, August 31 \\
\end{tabsched}

	\section{Unit 2: The Impact of Stories on Teaching and Learning}

\gentopic{The stories that we tell about ourselves and others influence the attitudes and values that we bring\\to our lives and learning experiences.}

\begin{tabsched}
	\tabdt & \weekthree \\
	\midrule
	\tabread & Between the World and Me (\url{http://theatln.tc/1J4PS2b}) \\
	& Rachel Dolezal, in Center of Storm, Is Defiant: 'I Identify as Black' \\
	& (\url{http://nyti.ms/1DAKIZQ}) \\
	& Inside Appalachia: WV Mine Wars, Red-Neck Folklore, \& More \\
	& (\url{http://bit.ly/1DAKOAI} up until the 20 minute mark, \\
	& although you are welcome to listen to the whole episode) \\
	\midrule
	\tabcheckin & \textbf{Re-Remembering Rednecks} Due Friday, September 4 \\ % (: What did I think and what have I learned?)
\end{tabsched}

\tabbreak

\specialweek{\laborday}

\begin{tabsched}
	\tabdt & \weekfour \\
	\midrule
	\tabread & Politics in the Classroom: How Much is Too Much? (\url{http://n.pr/1J4QaGw}) \\
	& How Mindfulness Can Defeat Racial Bias (\url{http://bit.ly/1UxGjL6}) \\
	& Teaching While White (\url{http://bit.ly/1IwsgAe})\\
	\midrule
	\tabperformance & \textbf{The Redneck Project} Due Monday, September 14 \\
\end{tabsched}

	\section{Unit 3: Your Work to Model Social and Emotional Approaches to Teaching and Learning}

\gentopic{A key part of the intentional effort involves modeling by educators of introspection,\\contemplative and deliberative practices, and identity work.}

\specialweek{\roshhashanah}

\begin{tabsched}
	\tabdt & \weekfive \\
	\midrule
	\tabread & Juvenile Injustice (\url{http://slate.me/1L13xVJ}) \\
	& In Classroom Discipline, a Soft Approach Is Harder Than It Looks \\
	& (\url{http://bit.ly/1IFXZg9}) \\
	& Fresh Starts for Hard to Like Students (\url{http://bit.ly/1N1GkVC}) \\
	& 'You are more than your mistakes': Teachers get at roots of bad behavior (\url{http://bit.ly/1Wayi0t}) \\
	& Sometimes Misbehavior Is Not What It Seems (\url{http://bit.ly/1PeON75}) \\
	\midrule
	\tabcheckin & \textbf{Practices Organizer} Due Friday, September 18 \\
\end{tabsched}

\tabbreak

\newpage

\specialweek{\yomkippur}

\begin{tabsched}
	\tabdt & \weeksix \\
	\midrule
	\tabread & How SEL and Mindfulness Can Work Together (\url{http://bit.ly/1DMIGFM}) \\
	& Shared Mindfulness: Building Supportive Relationships in the Classroom (\url{http://bit.ly/1f2n3Wc}) \\
	& Energy and Calm: Brain Breaks and Focused-Attention Practices \\
	& (\url{http://bit.ly/1NklTRE}) \\
	& 3 Steps to Increase Students' Self-Awareness: Identifying Strengths and Challenges (\url{http://bit.ly/1J4QXqR}) \\
	\midrule
	\tabperformance & \textbf{Contemplative Practices Tree} Due on Friday, September 25 \\
	& \textbf{Contemplative Practices Discussion Day} on Monday, September 28 \\
\end{tabsched}

	\section{Unit 4: Social and Emotional Teaching and Learning in the Wide World}

\gentopic{Social and emotional learning and teaching occurs within a broader social and political context.}

\begin{tabsched}
	\tabdt & \weekseven \\
	\midrule
	\tabread & Five Ways Teachers Can Limit the Fear of Creative Failure \\
	& (\url{http://bit.ly/1N1ST3k}) \\
	& Do we have the confidence to allow students to be playful learners? \\
	& (\url{http://bit.ly/1IXY5mT}) \\
	& Oopsigami (\url{http://bit.ly/1MeFggd}) \\
	& Using Origami to Foster a Growth Mindset (\url{http://bit.ly/1f2p44E}) \\
	\midrule
	\tabcheckin & \textbf{Activity Organizer} Due Friday, October 2 \\
\end{tabsched}

\tabbreak

\finisemester

\begin{tabsched}
	\tabdt & \weekeight \\
	\midrule
	\tabread & How to Integrate Social-Emotional Learning into Common Core \\
	& (\url{http://bit.ly/1IY0jTg}) \\
	& Low-Income Schools See Big Benefits in Teaching Mindfulness \\
	& (\url{http://bit.ly/1DAWyDd}) \\
	& The Language of Choice and Support (\url{http://bit.ly/1hskU8a}) \\
	& Believing in Students: The Power to Make a Difference (\url{http://bit.ly/1IwxgEZ}) \\
	\midrule
	\tabperformance & \textbf{Social-Emotional Activity Plan} Due by Monday, October 12 \\
\end{tabsched}

\end{document}