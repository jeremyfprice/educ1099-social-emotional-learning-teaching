%!TEX program = lualatex

\documentclass[aspectratio=169]{beamer}
%\documentclass[12pt,t]{beamer}
\beamertemplatenavigationsymbolsempty
\usetheme{default}
\usepackage{graphicx}
\graphicspath{{presentation2-img/}}
\setbeameroption{hide notes}
\setbeamertemplate{note page}[plain]
\setbeamerfont{frametitle}{size=\huge}

\usefonttheme{professionalfonts} % using non standard fonts for beamer
\usefonttheme{serif} % default family is serif
\usepackage{fontspec}
\setmainfont[Renderer=Basic, Numbers=OldStyle, Scale = 1.0]{Noto Serif}
\setsansfont[Renderer=Basic, Numbers=OldStyle, Scale = 1.0]{Noto Sans}
\setmonofont{Deja Vu Sans Mono}

\usepackage{color}
\usepackage{microtype}

\usepackage{booktabs}
\usepackage{array,multirow}

\usepackage{csquotes}
\MakeOuterQuote{"}


\usepackage{tikz}
\usetikzlibrary{shadows,calc}

% code adapted from http://tex.stackexchange.com/a/11483/3954

% some parameters for customization
\def\shadowshift{3pt,-3pt}
\def\shadowradius{6pt}
\colorlet{innercolor}{black!60}
\colorlet{outercolor}{gray!05}

% this draws a shadow under a rectangle node
\newcommand\drawshadow[1]{
	\begin{pgfonlayer}{shadow}
		\shade[outercolor,inner color=innercolor,outer color=outercolor] ($(#1.south west)+(\shadowshift)+(\shadowradius/2,\shadowradius/2)$) circle (\shadowradius);
		\shade[outercolor,inner color=innercolor,outer color=outercolor] ($(#1.north west)+(\shadowshift)+(\shadowradius/2,-\shadowradius/2)$) circle (\shadowradius);
		\shade[outercolor,inner color=innercolor,outer color=outercolor] ($(#1.south east)+(\shadowshift)+(-\shadowradius/2,\shadowradius/2)$) circle (\shadowradius);
		\shade[outercolor,inner color=innercolor,outer color=outercolor] ($(#1.north east)+(\shadowshift)+(-\shadowradius/2,-\shadowradius/2)$) circle (\shadowradius);
		\shade[top color=innercolor,bottom color=outercolor] ($(#1.south west)+(\shadowshift)+(\shadowradius/2,-\shadowradius/2)$) rectangle ($(#1.south east)+(\shadowshift)+(-\shadowradius/2,\shadowradius/2)$);
		\shade[left color=innercolor,right color=outercolor] ($(#1.south east)+(\shadowshift)+(-\shadowradius/2,\shadowradius/2)$) rectangle ($(#1.north east)+(\shadowshift)+(\shadowradius/2,-\shadowradius/2)$);
		\shade[bottom color=innercolor,top color=outercolor] ($(#1.north west)+(\shadowshift)+(\shadowradius/2,-\shadowradius/2)$) rectangle ($(#1.north east)+(\shadowshift)+(-\shadowradius/2,\shadowradius/2)$);
		\shade[outercolor,right color=innercolor,left color=outercolor] ($(#1.south west)+(\shadowshift)+(-\shadowradius/2,\shadowradius/2)$) rectangle ($(#1.north west)+(\shadowshift)+(\shadowradius/2,-\shadowradius/2)$);
		\filldraw ($(#1.south west)+(\shadowshift)+(\shadowradius/2,\shadowradius/2)$) rectangle ($(#1.north east)+(\shadowshift)-(\shadowradius/2,\shadowradius/2)$);
	\end{pgfonlayer} }
% create a shadow layer, so that we don't need to worry about overdrawing other things
\pgfdeclarelayer{shadow}
\pgfsetlayers{shadow,main}
\newsavebox\mybox
\newlength\mylen
\newcommand\shadowimage[2][]{%
	\setbox0=\hbox{\includegraphics[#1]{#2}}
	\setlength\mylen{\wd0}
	\ifnum\mylen<\ht0
		\setlength\mylen{\ht0}
	\fi
	\divide \mylen by 120
	\def\shadowshift{\mylen,-\mylen}
	\def\shadowradius{\the\dimexpr\mylen+\mylen+\mylen\relax}
	\begin{tikzpicture}
		\node[anchor=south west,inner sep=0] (image) at (0,0) {\includegraphics[#1]{#2}};
		\drawshadow{image}
	\end{tikzpicture}}

\definecolor{bgcolor}{RGB}{255,255,243}
\definecolor{FSUred}{RGB}{134,0,56}
\definecolor{fgblack}{RGB}{0,0,0} % I can't seem to figure out a way around this...

\setbeamercolor{background canvas}{bg=bgcolor}
\setbeamercolor{titlelike}{fg=fgblack}
\setbeamercolor{subtitle}{fg=fgblack}
\setbeamercolor{frametitle}{fg=FSUred}
\setbeamercolor{subtitle}{fg=FSUred}

\author{EDUC1199}
\date{19 August 2015}
\title{\Huge\textsf{A FRAMEWORK}\normalsize}
\subtitle{For Social and Emotional Approaches to Teaching and Learning}

\newcommand{\tBold}[1]{\textcolor{FSUred}{\textbf{#1}}}

\begin{document}
	\begin{frame}

		\vfill

		\maketitle

		\vfill

	\end{frame}

	\begin{frame}
		\frametitle{\textsf{A Review of the Framework}}
		\footnotesize\begin{tabular}{ccl}
			\toprule
			\multirow{9}{*}{\parbox[l]{3cm}{\raggedright\large New Haven Social Development Curriculum \footnotesize}} & \multirow{3}{*}{\normalsize Skills \footnotesize} & Self-Management \\
			 & & Problem Solving and Decision Making \\
			 & & Communication \\
			\cline{2-3}
			 & \multirow{3}{*}{\normalsize Attitudes and Values \footnotesize} & About Self \\
			 & & About Others \\
			 & & About Tasks \\
			 \cline{2-3}
			 & \multirow{3}{*}{\normalsize Content \footnotesize} & Self/Health \\
			 & & Relationships \\
			 & & School/Community \\
			 \bottomrule
		\end{tabular}
	\end{frame}

	\begin{frame}
		\frametitle{\textsf{How is the framework represented?}}
		\footnotesize\begin{tabular}{ccl}
			\toprule
			\multirow{9}{*}{\parbox[l]{3cm}{\raggedright\large New Haven Social Development Curriculum \footnotesize}} & \multirow{3}{*}{\normalsize Skills \footnotesize} & Self-Management \\
			 & & Problem Solving and Decision Making \\
			 & & Communication \\
			\cline{2-3}
			 & \multirow{3}{*}{\normalsize Attitudes and Values \footnotesize} & About Self \\
			 & & About Others \\
			 & & About Tasks \\
			 \cline{2-3}
			 & \multirow{3}{*}{\normalsize Content \footnotesize} & Self/Health \\
			 & & Relationships \\
			 & & School/Community \\
			 \bottomrule
		\end{tabular}
	\end{frame}

	\begin{frame}
		\frametitle{\textsf{What emotions do}}
		{\Large}“{\ldots}[E]motions \textit{organize} \textendash rather than disrupt \textendash our social lives. Studies have found, for example, that emotions \textit{structure} (not just color) such disparate social interactions as attachment between parents and children, sibling conflicts, flirtations between young courters and negotiations between rivals.”
		\vspace{2em}
		\begin{flushright}
			\tiny (Keltner \& Ekman, 2015)
		\end{flushright}
	\end{frame}

	\begin{frame}
		\frametitle{\textsf{Recognizing the full spectrum of emotions}}
		{\Large}“‘Inside Out’ offers a new approach to sadness. Its central insight: Embrace sadness, let it unfold, engage patiently with a preteen's emotional struggles. Sadness will clarify what has been lost (childhood) and move the family toward what is to be gained: the foundations of new identities, for children and parents alike.”
		\vspace{2em}
		\begin{flushright}
			\tiny (Keltner \& Ekman, 2015)
		\end{flushright}
	\end{frame}

	\begin{frame}
		\frametitle{\textsf{Recognizing the full spectrum of emotions}}
		\begin{columns}[c] % align columns
			\begin{column}{.48\textwidth}
				{\Large}“Toward the end of the movie, Joy does what some researchers now consider to be the healthiest method for working with emotions: Instead of avoiding or denying Sadness, Joy accepts Sadness for who she is, realizing that she is an important part of Riley’s emotional life.”
				\vspace{2em}
				\begin{flushright}
					\tiny (Marsh \& Zakrzewski, 2015)
				\end{flushright}
			\end{column}%
			\hfill
			\begin{column}{.48\textwidth}
				\centering\noindent\shadowimage[width=0.9\textwidth]{Inside_Out_Joy_and_Sadness.jpg}
			\end{column}
		\end{columns}
	\end{frame}

	\begin{frame}
		\frametitle{\textsf{Recognizing the full spectrum of emotions}}
		\begin{columns}[c] % align columns
			\begin{column}{.48\textwidth}
				{\Large}“Emotion experts call this ‘mindfully embracing’ an emotion. What does that mean? Rather than getting caught up in the drama of an emotional reaction, a mindful person kindly observes the emotion without judging it as the right or wrong way to feel in a given situation, creating space to choose a healthy response.”
				\vspace{2em}
				\begin{flushright}
					\tiny (Marsh \& Zakrzewski, 2015)
				\end{flushright}
			\end{column}%
			\hfill
			\begin{column}{.48\textwidth}
				\centering\noindent\shadowimage[width=0.9\textwidth]{Inside_Out_Joy_and_Sadness.jpg}
			\end{column}
		\end{columns}
	\end{frame}

\end{document}