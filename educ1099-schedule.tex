%!TEX program = luatex

\documentclass{article}
\usepackage[margin=1in]{geometry}
\usepackage{fontawesome}
\usepackage{fontspec}
\setmainfont[Renderer=Basic, Numbers=OldStyle, Scale = 1.0]{Noto Serif}
\setsansfont[Renderer=Basic, Numbers=OldStyle, Scale = 1.0]{Noto Sans}
\setmonofont{Deja Vu Sans Mono}
% The following package makes prettier tables.  We're all about the bling!
\usepackage{booktabs}

\newcommand{\gentopic}[1]{\begin{center}\faKey\hspace{1em}\textsf{#1}\end{center}}
\newcommand{\tabread}{\faBook\hspace{1em}\textit{Readings}}
\newcommand{\tabperformance}{\faTasks\medspace\textit{Performances}}
\newcommand{\tabdt}{\faCalendar\hspace{1em}\textit{Week Of}}
\newcommand{\tabcheckin}{\raggedright\faCheckSquareO\hspace{1em}\textit{Check-In Performances}}
\newcommand{\tabbreak}{\begin{center}\faAsterisk\faAsterisk\faAsterisk\\\end{center}}
\newcommand{\specialweek}[1]{\begin{center}\textbf{\faBullhorn\hspace{1em} Special Week: #1 \hspace{1em}\faBullhorn}\end{center}}
\newenvironment{tabsched}
	{\small
	\begin{tabular}{p{1.5in}p{4.5in}}
	\midrule}
	{\midrule
	\end{tabular}
	\normalsize}

\newenvironment{specweek}
	{\begin{center}
		\fontseries{b} \faBullhorn \medspace Special Week: }
		{\medspace \faBullhorn \fontseries{m}
	\end{center}}

\newcommand{\weekone}{August 17-21}
\newcommand{\weektwo}{August 24-28}
\newcommand{\weekthree}{August 31-September 4}
\newcommand{\weekfour}{September 7-11}
\newcommand{\weekfive}{September 14-18}
\newcommand{\weeksix}{September 21-25}
\newcommand{\weekseven}{September 28-October 2}
\newcommand{\weekeight}{October 5-9}
\newcommand{\laborday}{Labor Day on Monday, September 7 (no class)}
\newcommand{\roshhashanah}{Rosh Hashanah on Monday, September 14 (class via VoiceThread)}
\newcommand{\yomkippur}{Yom Kippur on Wednesday, September 23 (class via VoiceThread)}
\newcommand{\finisemester}{\begin{center}\large\textbf{\faFlagCheckered Last Day of Class, Wednesday, October 7 \faFlagCheckered}\normalsize\end{center}}
\newcommand{\url}[1]{\footnotesize\texttt{#1}\normalsize}
\newcommand{\listmon}{\item[\large\textsf{Monday}\normalsize] \hfill \\}
\newcommand{\listwed}{\item[\large\textsf{Wednesday}\normalsize] \hfill \\}
\newenvironment{daywu}
	{\underline{\textbf{Warm-Up:}} \hfill \\
	\begin{itemize}}
	{\end{itemize}}
\newenvironment{dayact}
	{\underline{\textbf{Activities:}} \hfill \\
	\begin{itemize}}
	{\end{itemize}}
\newenvironment{dayref}
	{\underline{\textbf{Reflection:}} \hfill \\
	\begin{itemize}}
	{\end{itemize}}
\newenvironment{weeksched}
	{\noindent
	\begin{description}}
	{\end{description}
	\newpage}

\begin{document}
\section{Introducing Social and Emotional Approaches to Teaching and Learning}

\gentopic{Values and attitudes influence the educational process, are brought in by both students and educators, and can be developed through intentional efforts.}


\begin{tabsched}
	\tabdt & \weekone \\
	\midrule
	\tabread & The Need for Social and Emotional Learning (\url{http://bit.ly/1WauXyA}) \\
	& Why Emotional Learning May Be As Important As The ABCs \\
	& (\url{http://n.pr/1OZuvO1}) \\
	& How to Be Emotionally Intelligent (\url{\texttt{http://nyti.ms/1hrH6it}}) \\
	& Nonacademic Skills are Key To Success. But What Should We Call Them?  (\url{http://n.pr/1PeOtFh}) \\
	\midrule
	\tabcheckin & \textbf{What Do I Notice?} Due Friday, August 21 \\
\end{tabsched}
\begin{weeksched}

\listmon
\begin{daywu}
	\item Purposeful First Day Exercise
		\begin{itemize}
			\item Name on one side
			\item What kind of educator/person do you want to be?
			\item What kind of impact do you want to have?
		\end{itemize}
	\item Pair-Share, and then share with class. Additional question: How do you want this class to help you with these answers?
\end{daywu}
\begin{dayact}
	\item Discuss brief overview of class and review syllabus
	\item Read poem, discuss in hevrutah
	\begin{itemize}
		\item Why is "ideal" important?
		\item How does this poem relate to learning?
		\item How does this poem relate to knowing?
	\end{itemize}
\end{dayact}
\begin{dayref}
	\item What's one thing you've found surprising today in class?
\end{dayref}

\listwed
\begin{daywu}
	\item Second-Day Graffiti
\end{daywu}
\begin{dayact}
	\item Pause for questions about the course, requirements, or due dates
	\item "Lecture" on Framework
\end{dayact}
\begin{dayref}
	\item Anonymous remaining questions on note cards
\end{dayref}
\end{weeksched}
\tabbreak
\begin{tabsched}
	\tabdt & \weektwo \\
	\midrule
	\tabread & How Emotions Affect Learning (\url{http://bit.ly/1TivmL9})\\
	& The Science of Inside Out (\url{http://nyti.ms/1TivSsn})\\
	& Four Lessons from Inside Out to Discuss With Kids (\url{http://bit.ly/1KY4ONg})\\
	\midrule
	\tabperformance & \textbf{48-Hour Social-Emotional Journal} Due Monday, August 31 \\
\end{tabsched}
\begin{weeksched}

\listmon
\begin{daywu}
	\item Go around room, identify the prominent emotion you are bringing in.
\end{daywu}
\begin{dayact}
	\item Review Framework
	\item \textit{Matzli'ach Li} reading
	\item Three groups and distribute books
	\item How is the framework represented in the books?
	\item Write on chart paper, then share out
	\item Watch each video (\url{http://bit.ly/1h6UKY0} and \url{http://bit.ly/1JdVx0B}), students take notes on how represented, pair-share
\end{dayact}
\begin{dayref}
	\item What's one thing you've found surprising today in class?
\end{dayref}

\listwed
\begin{daywu}
	\item Watch School Bully video (\url{http://bit.ly/1EcBIda}), and discuss what's going on: how does it relate to what we've been discussing?
\end{daywu}
\begin{dayact}
	\item Show both videos (https://youtu.be/xNY0AAUtH3g and https://www.youtube.com/watch?v=gAMbkJk6gnE [six minutes and after])
	\item Link back to readings, release of neuropeptides and structures in the brain, physicalness of the process
	\item Teams of 2 and 3, create storyboard based on what they've learned for a PSA on the connections between emotions and learning
	\item Share briefly
\end{dayact}
\begin{dayref}
	\item What's one thing you've found surprising today in class?
\end{dayref}
\end{weeksched}

\tabbreak

\section{Unit 2: The Impact of Stories on Teaching and Learning}

\gentopic{The stories that we tell about ourselves and others influence the attitudes and values that we bring\\to our lives and learning experiences.}

\begin{tabsched}
	\tabdt & \weekthree \\
	\midrule
	\tabread & Between the World and Me (\url{http://theatln.tc/1J4PS2b}) \\
	& Rachel Dolezal, in Center of Storm, Is Defiant: 'I Identify as Black' \\
	& (\url{http://nyti.ms/1DAKIZQ}) \\
	& Inside Appalachia: WV Mine Wars, Red-Neck Folklore, \& More \\
	& (\url{http://bit.ly/1DAKOAI} up until the 20 minute mark, \\
	& although you are welcome to listen to the whole episode) \\
	\midrule
	\tabcheckin & \textbf{Re-Remembering Rednecks} Due Friday, September 4 \\ % (: What did I think and what have I learned?)
\end{tabsched}

\begin{weeksched}

\listmon
\begin{daywu}
	\item Read Dialogue Guidelines
	\item Close-Far Mindful Examination of \textit{The Problem We All Live With} (adapted from Zajonc)
	\begin{itemize}
		\item Take in the painting as a whole for one minute, without judging and evaluating your thoughts, just noticing
		\item Focus on one detail, big or small, without judging and evaluating your thoughts, just noticing
		\item Start noticing the words that you associate with the detail, picture these words, arrange them around the detail, hold on to these words
		\item Take in the painting as a whole again
		\item Write down these words in the pattern you arranged them in
		\item Share and discuss, ask for volunteers first
	\end{itemize}
\end{daywu}
\begin{dayact}
	\item Show Michael Brown's mother video, the part about putting a black boy through high school
	\item What does this say to you? How can you connect this with the framework we have been discussing, the idea of mind-body, neurological pathways, and the readings?
\end{dayact}
\begin{dayref}
	\item What's one thing you've found surprising today in class?
\end{dayref}

\listwed
\begin{daywu}
	\item Read Dialogue Guidelines
	\item Beholding Exercise with \textit{The Problem We All Live With} (adopted from Barbezat \& Bush)
	\begin{itemize}
		\item Take in the painting as a whole for one minute, without judging and evaluating your thoughts, just noticing
		\item Distribute question note cards, have them consider the questions while beholding the painting
		\item Come back together, and share and discuss, ask for volunteers first
	\end{itemize}
\end{daywu}
\begin{dayact}
	\item Make shift from race to geography and social class (SES)
	\item Ask for word associations with "Red Neck" and write on board
	\item How does that make you feel?
	\item Show Wild and Wonderful Whites of WV trailer first, write down some words that come to mind after watching it, share and discuss. Include some discussion about how stories influence lives and the way we do things and experience things.
	\item Show Behind the WWWWV video, ask if it changes your minds about your words or reinforce them
	\item What does this say to you? How can you connect this with the framework we have been discussing, the idea of mind-body, neurological pathways, and the readings?
	\item Share information about history of Red Neck, Blair Mountain
	\item Share some word associations of how you'd like to see Red Necks
	\item Share project
\end{dayact}
\begin{dayref}
	\item What's one thing you've found surprising today in class?
\end{dayref}
\end{weeksched}

\end{document}